\documentclass[a4paper,12pt]{article}
\usepackage[backend=bibtex]{biblatex}
\usepackage{siunitx}
\usepackage{microtype}
\usepackage{float}
\usepackage{graphicx}
\usepackage{bm}
\usepackage{amsmath}
\usepackage{parskip}
\usepackage{longtable}
\usepackage{enumerate}
\graphicspath{}
\usepackage{tikz}
\usetikzlibrary{trees}
\usepackage{amssymb}
\usepackage{hyperref}
\usetikzlibrary{decorations.pathmorphing}
\usetikzlibrary{decorations.markings}
\usepackage{cleveref}
\usepackage{titlesec}
\newcommand{\sectionbreak}{\clearpage}
\title{Combinations and Permutations}
\date{\today}
\author{Anas Syed}
\tikzstyle{level 1}=[level distance=3.5cm, sibling distance=2.5cm]
\tikzstyle{level 2}=[level distance=3.5cm, sibling distance=2cm]
\tikzstyle{bag} = [text width=4em, text centered]
\tikzstyle{end} = [circle, minimum width=3pt,fill, inner sep=0pt]

\begin{document}
\maketitle
\tableofcontents
\newpage
\section{Introduction}
Define the set $A$ where $|A|=n$. Obviously, every element is unique, by the definition of a set.
\section{Permutations}
Order matters.
\subsection{With repetition}
\subsubsection{Problem}
Work out how many ways there are of choosing $r$ elements from $A$, and we can choose the same element multiple times.

\subsubsection{Solution}
The first time we choose an element, we have $n$ choices. It is the same for all $r$ choices we make. The first choice adds $n$ possible choices. Then for 

%The situation can be seen from \Cref{fig:tree}. The answer to the problem is the number of branches in this tree. If each choice creates $n$ branches, 

%\begin{figure}[H]
  %\centering
  %\begin{tikzpicture}[grow=right]
    %\centering
    %\node[bag] {First choice}
    %child {
      %node[bag] {Second choice}        
      %child {
        %edge from parent
        %node[above] {$\vdots$}
      %}
      %child {
        %edge from parent
        %node[below] {$\vdots$}
      %}
      %edge from parent 
      %node[above] {$\vdots$}
    %}
    %child {
      %node[bag] {Second choice}        
      %child {
        %edge from parent
        %node[above] {$\vdots$}
      %}
      %child {
        %edge from parent
        %node[below] {$\vdots$}
      %}
      %edge from parent 
      %node[below] {$\vdots$}
    %};
  %\end{tikzpicture}
  %\caption{Tree diagram showing choices}
  %\label{fig:tree}
%\end{figure}

\end{document}
