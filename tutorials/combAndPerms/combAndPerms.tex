\documentclass[a4paper,12pt]{article}
\newtheorem{theorem}{Theorem}
\newtheorem{definition}{Definition}
\newtheorem{lemma}{Lemma}
\newtheorem{proof}{Proof}
\usepackage{siunitx}
\usepackage{microtype}
\usepackage{float}
\usepackage{graphicx}
\usepackage{bm}
\usepackage{amsmath}
\usepackage{parskip}
\usepackage{longtable}
\usepackage{enumerate}
\graphicspath{}
\usepackage{tikz}
\usetikzlibrary{trees}
\usepackage{amssymb}
\usepackage{hyperref}
\usepackage{cleveref}
\usepackage{titlesec}
\title{Combinations and Permutations}
\date{\today}
\author{Anas Syed}
\begin{document}
\maketitle
\tableofcontents
\newpage
\section{Introduction}
Define the set $A$ where $|A|=n$. Obviously, every element is unique, by the definition of a set.

\subsection{Product rule of counting}
We will assume the product rule of counting.
\begin{theorem}
  Say we have a collection of $r$ sets, $A_1$, $A_2$, to $A_r$ with respective cardinalities of $n_1$, $n_2$, to $n_r$. Suppose we have to successively choose an element from each set in order. A particular sequence of items chosen from the $r$ sets is a \emph{composite outcome}. The number of distinct composite items, where order matters, is
  \begin{equation}
    \prod_{i=1}^r n_i
  \end{equation}
  \label{thm:productRule}
\end{theorem}

\section{Permutations}
Order matters.
\subsection{With repetition}
\label{subsec:permWithRep}
\subsubsection{Problem}
Work out how many ways there are of choosing $r$ elements from $A$, and we can choose the same element multiple times.

\subsubsection{Solution}
The first time we choose an element, we have $n$ choices. It is the same for all $r$ choices we make. By \cref{thm:productRule}, the total number of outcomes is $n^r$.

\subsection{Without repetition}
\subsubsection{Problem}
Work out how many ways there are of choosing $r$ elements from $A$, and we cannot choose the same element twice.

\subsubsection{Solution}
At the first step, we will have $n$ choices. At the successive step, if it exists, we will have $n-1$ choices. This continues until there are $n-r+1$ choices. By \cref{thm:productRule}, the answer is

\begin{equation}
  \underbrace{n \times (n-1) \times \ldots \times (n-r+1)}_{\text{$r$ terms being multiplied}} = \frac{n!}{(n-r)!}
\end{equation}

\section{Combinations}
Order doesn't matter.
\subsection{Without repetition}
\label{subsec:combWithoutRep}
\subsubsection{Problem}
Say we can create sets by choosing $r$ elements from $A$, and we cannot choose the same element twice. Work out how many distinct sets can be created in this way.

\subsubsection{Solution}
The set of the distinct sets (call this $B$) that can be created in this way is a subset of the distinct sequences that can be created in \cref{subsec:permWithRep} because order doesn't matter.

The number of ways of organising $r$ items is $r!$, by \cref{thm:productRule}. Therefore, there are $r!$ different ordered sequences (call these $\{s_1, s_2, \ldots, s_{r!}\}$) in the problem in \cref{subsec:permWithRep} for every unordered sequence in $B$.

Also, none of the sequences $s_1$ to $s_{r!}$ are the same, i.e.\

\begin{equation}
  s_i \neq s_j\ \forall\ i, j \in \{1, 2, \ldots, r!\}\ \mbox{s.t.}\ i \neq j
\end{equation}

This means that
\begin{equation}
  |B| = \frac{n!}{(n-r)!} \times \frac{1}{r!} = \binom{n!}{r!(n-r)!}
\end{equation}

\subsection{With repetition}
\subsubsection{Problem}
Work out how many ways there are of choosing $r$ elements from $A$, but we can choose the same element multiple times. The order of the sequence does not matter. This is a harder problem - we cannot simply adapt the solution equivalent permutation problem like we did in \cref{subsec:combWithoutRep} because we need to remove the solutions which have a different order \emph{but} swapping the order of two items which are the same (have been repeated) makes no difference.

\subsubsection{Solution}
Say we choose $r$ elements from $A$, and we can choose the same element multiple times. If we represent an element by $\circ$, we could then arrange the chosen elements with dividers (represented by $|$) into separate bins. This leads to a unique representation of an outcome for this problem (since the order doesn't matter, swapping instances of a circle does not make a difference).

An example representation of a unique outcome would be

\begin{equation}
  \circ | \circ \circ | | \circ \circ \circ | |
\end{equation}

From the above outcome, 1 item was chosen to be the 1st element of $A$, 2 items were chosen to be the 2nd, 0 items were chosen to be the 3rd, 3 items were chosen to be the 4th and no items were chosen to be the 5th or 6th elements of $A$.

There must be exactly $r$ circles and $n-1$ bars since we are choosing $r$ items from $n$ categories, and $n-1$ bars can separate $n$ categories. Furthermore, and arbitrary arrangement of the bars and circles leads to a valid, unique outcome for the problem. In addition, we can represent any possible outcome in this way.

Therefore, we can say that the answer is the same as the number of ways of arranging these bars and circles. There are $r+(n-1)$ positions in which we can place either a bar or a circle. Once we have placed all the circles, the position of the bars is also decided. This is like the problem in \cref{subsec:combWithoutRep}. It is like choosing $r$ items from $r+n-1$ positions. This means that the answer is $\binom{r+n-1}{r}$.

Obviously, we can also choose the bars first, and this fixes the position of the circles. This gives the answer $\binom{r+n-1}{n-1}$, which is the same as $\binom{r+n-1}{r}$. The final answer is

\begin{equation}
  \binom{r+n-1}{r} = \binom{r+n-1}{n-1} = \frac{(r+n-1)!}{k!(n-1)!}
\end{equation}

\end{document}
