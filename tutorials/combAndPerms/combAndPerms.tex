\documentclass[a4paper,12pt]{article}
\newtheorem{theorem}{Theorem}
\newtheorem{definition}{Definition}
\newtheorem{lemma}{Lemma}
\newtheorem{proof}{Proof}
\usepackage[backend=bibtex]{biblatex}
\usepackage{siunitx}
\usepackage{microtype}
\usepackage{float}
\usepackage{graphicx}
\usepackage{bm}
\usepackage{amsmath}
\usepackage{parskip}
\usepackage{longtable}
\usepackage{enumerate}
\graphicspath{}
\usepackage{tikz}
\usetikzlibrary{trees}
\usepackage{amssymb}
\usepackage{hyperref}
\usepackage{cleveref}
\usepackage{titlesec}
\newcommand{\sectionbreak}{\clearpage}
\title{Combinations and Permutations}
\date{\today}
\author{Anas Syed}
\begin{document}
\maketitle
\tableofcontents
\newpage
\section{Introduction}
Define the set $A$ where $|A|=n$. Obviously, every element is unique, by the definition of a set.

\subsection{Product rule of counting}
We will assume the product rule of counting.
\begin{theorem}
  Say we have a collection of $r$ sets, $A_1$, $A_2$, to $A_r$ with respective cardinalities of $n_1$, $n_2$, to $n_r$. Suppose we have to successively choose an element from each set in order. A particular sequence of items chosen from the $r$ sets is a \emph{composite outcome}. The number of distinct composite items, where order matters, is
  \begin{equation}
    \prod_{i=1}^r n_i
  \end{equation}
  \label{thm:productRule}
\end{theorem}

\section{Permutations}
Order matters.
\subsection{With repetition}
\label{subsec:permWithRep}
\subsubsection{Problem}
Work out how many ways there are of choosing $r$ elements from $A$, and we can choose the same element multiple times.

\subsubsection{Solution}
The first time we choose an element, we have $n$ choices. It is the same for all $r$ choices we make. By \cref{thm:productRule}, the total number of outcomes is $n^r$.

\subsection{Without repetition}
\subsubsection{Problem}
Work out how many ways there are of choosing $r$ elements from $A$, and we cannot choose the same element twice.

\subsubsection{Solution}
At the first step, we will have $n$ choices. At the successive step, if it exists, we will have $n-1$ choices. This continues until there are $n-r+1$ choices. By \cref{thm:productRule}, the answer is

\begin{equation}
   n \times (n-1) \times \ldots \times (n-r+1) = \frac{n!}{(n-r)!}
\end{equation}

\section{Combinations}
Order doesn't matter.
\subsection{Without repetition}
\subsubsection{Problem}
Say we can create sets by choosing $r$ elements from $A$, and we cannot choose the same element twice. Work out how many distinct sets can be created in this way.

\subsubsection{Solution}
The set of the distinct sets (call this $B$) that can be created in this way is a subset of the distinct sequences that can be created in \cref{subsec:permWithRep} because order doesn't matter.

The number of ways of organising $r$ items is $r!$, by \cref{thm:productRule}. Therefore, there are $r!$ different sequences in the problem in \cref{subsec:permWithRep} for every unordered sequence in $B$. This means that
\begin{equation}
  |B| = \frac{n!}{(n-r)!} \times \frac{1}{r!}
\end{equation}

\subsection{With repetition}
\subsubsection{Problem}
Work out how many ways there are of choosing $r$ elements from $A$, but we cannot choose the same element multiple times. The order of the sequence does not matter.

\subsubsection{Solution}
The first time we choose an element, we have $n$ choices. It is the same for all $r$ choices we make. By \cref{thm:productRule}, the total number of outcomes is $n^r$.


\end{document}
